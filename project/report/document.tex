\documentclass[12pt]{article}

\usepackage{sbc-template}

\usepackage{graphicx,url}
\usepackage{csvsimple}
\usepackage{subfigure}

\usepackage{amssymb}% http://ctan.org/pkg/amssymb
\usepackage{pifont}% http://ctan.org/pkg/pifont
\newcommand{\cmark}{\ding{51}}%
\newcommand{\xmark}{\ding{55}}%

\usepackage{multirow}
\usepackage[brazil]{babel}    
\usepackage[utf8]{inputenc}
 
\sloppy

\title{Simulação de \textit{tracking} da dinâmica dos elétrons em um acelerador
síncrotron}

\author{Gustavo Ciotto Pinton\inst{1} - RA117136 }


\address{Instituto de Computação -- Universidade Estadual de Campinas
(UNICAMP)\\
  Av. Albert Einstein, 1251, Cidade Universitária, Campinas/SP \\
  Brasil, CEP 13083-852, Fone: [19] 3521-5838
  \email{gustavociotto@gmail.com}
}

\begin{document} 

\maketitle

\begin{abstract}
This report presents and describes the main aspects of the development of a
parallel solution for a sequential algorithm, whose purpose is simulating
the dynamics (modeled by a 6-element vector \(X\)) of electrons that
run through a synchrotron accelerator. In general, in this algorithm, the whole
extension of the accelerator is divided into several elements capable of
changing the state of each electron in a certain way, and it is sought to
determine which initial conditions of \(X\) will meet some conditions after a
certain number of revolutions. All the parallelization was based on GPUs and the
CUDA library, obtaining gains of performance of the order of 3000\% in relation
to the serial execution.
\end{abstract}
     
\begin{resumo} 
Este relatório apresenta e descreve os principais aspectos do
desenvolvimento de uma solução paralela para um algoritmo sequencial de
simulação da dinâmica do movimento dos elétrons (modelado por um vetor \(X\) de
6 posições) que percorrem um acelerador síncrotron. De maneira geral, neste
algoritmo, divide-se toda a extensão do acelerador em diversos elementos capazes de alterar o estado de cada életron de
uma determinada maneira, e busca-se quais condições iniciais de \(X\) atenderão
algumas condições após um determinado número de voltas. Toda a paralelização foi
baseada em GPUs e na biblioteca CUDA, obtendo-se ganhos de performance da ordem
de 3000\% em relação à execução serial.
\end{resumo}


\section{Introdução}

Aceleradores de partículas síncrotron são constituídos, ao longo de todas suas
circunferências, de atuadores, tais como dipolos, quadrupólos e sextupólos,
responsáveis por modificar a direção dos elétrons através da imposição de campos
magnéticos. Pode-se modelar o estado de um elétron na entrada do acelerador por
um vetor no espaço de fase constituído por 6 elementos \( X = (x_1, x_2, \ldots,
x_6) \). A cada passagem por um determinado atuador, esse vetor é modificado por
um mapa \(F_n(X)\) que depende, evidentemente, do tipo de atuador e de seus
parâmetros. A simulação de \textit{tracking} da dinâmica de movimento dos
elétrons consiste, portanto, em determinar quais vetores iniciais \(X_i\) neste
espaço de fase correspondem à órbitas estáveis após o percurso de \(N\) voltas
pelo acelerador. Um vetor \(X_i\) pode ser considerado uma
órbita estável somente se, ao fim de \(N\) voltas completas, as posições \(x_1,
\ldots, x_6 \) são inferiores às constantes \(C_1, \ldots, C_6\). A explicação
dos significados físicos de cada uma dessas posições não faz parte do escopo
deste artigo.

Durante os experimentos, o acelerador foi modelado por um número
constante \(M = 10000\) de atuadores e a cardinalidade do conjunto de vetores
iniciais testados foi igual a \(I = 10000\). Além disso, variou-se o número de
voltas \(N\) entre 10 e 10000, de modo a avaliar o ganho de performance em
função do número total de iterações desenvolvidas.

Aproveitando-se a alta densidade de \textit{cores} de arquitetura SIMD
encontrada nas GPUs atuais, a paralelização desta simulação consistiu em
distribuir o cálculo de cada condição inicial \(X_i\) em uma \textit{thread}
distinta executada pela GPU, de modo a maximizar a quantidade de
\textit{threads} que rodam simultaneamente. A implementação foi a baseada no
\textit{toolkit CUDA}, desenvolvido para as placas de vídeo da \textit{NVIDIA}.
Como todos os vetores \(X_i\) passarão pelos mesmos atuadores na mesma ordem e
não farão acessos a posições não sequenciais de memória, a paralelização deste
algoritmo não foi afetado pelos principais fatores que degradam a performance
neste tipo de arquitetura, sendo eles, respectivamente, \textit{branching
divergence} e acessos \textit{non-coalesced} à memória.

As próximas seções são dedicadas aos processos de implementação e aos resultados
obtidos.


\section {Análise da execução sequencial}

Com o intuito de identificar qual trecho do programa possui o maior potencial de
paralelização, isto é, onde a execução permanece por mais tempo, a ferramenta
\texttt{gprof} foi utilizada.

\section{Resultados}



\section{Conclusões}



\bibliographystyle{sbc}
\bibliography{sbc-template}

\end{document}
